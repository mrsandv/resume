\documentclass[letterpaper]{article}
\usepackage[utf8]{inputenc}
\usepackage[spanish]{babel}
\usepackage{tgpagella}
\usepackage{geometry}
\geometry{
  letterpaper,
  left=20mm,
  %right=
  %bottom=
  top=15mm}
\pagestyle{empty}
\usepackage{fontawesome5}

\newlength{\iconwidth}
\setlength{\iconwidth}{1.2em}
\usepackage{enumitem}
\usepackage[dvipsnames]{xcolor}

\definecolor{lavender}{HTML}{8E7CC3}
\definecolor{darklavender}{HTML}{674EA7}
\usepackage[colorlinks=true, linkcolor=blue, urlcolor=darklavender]{hyperref}

\usepackage{titlesec}

\titleformat{\section}
  {\color{darklavender}\Large}
  {}{0em}{}[{\color{darklavender}\titlerule[1pt]}]

\newlength{\spacebox}
\settowidth{\spacebox}{123456789}

\newcommand{\work}[4]{
\noindent  \textbf{#1}
\hfill
\mbox{
    \parbox{10em}{%
    \centering\textbf{#2}}} \par
\noindent \textit{#3} \par
\vspace*{0.5em}
\noindent\hangindent=2em\hangafter=0 \small #4
\normalsize \par}

\newcommand{\skill}[2]{
\noindent\hangindent=2em\hangafter=0
\parbox{3\spacebox}{
\textsc{#1}}
#2 \par
\vspace{0.5em}
}

\newcommand{\lan}[2]{
\noindent\hangindent=2em\hangafter=0
\parbox{\spacebox}{%
    \textbf{#1}}
    #2 \par}

\newcommand{\education}[4]{
\noindent  \textbf{#1}
    \hfill
    \mbox{
        \parbox{10em}{%
        \centering\textbf{#2}}} \par
    \noindent \textit{#3} \par
    \vspace*{0.5em}
    \noindent\hangindent=2em\hangafter=0 \small #4
\normalsize \par}

\begin{document}
\thispagestyle{empty}
\noindent
\textbf{\Huge \textcolor{darklavender}{Marco Antonio} \textcolor{lavender}{Sandoval Espinosa}}\par

\begin{flushleft}

\end{flushleft}\par

\noindent
\begin{itemize}[itemsep=0em]
\item[\faAt] \href{mailto:mrsandv@pm.me}{mrsandv@pm.me}
\item[\faLinkedin]\href{https://linkedin.com/in/mrsan}{linkedin.com/in/mrsan}
\item[\faPhone] \href{tel:+525580367317}{+52 55 803 673 17}
\item[\faMapMarker] \textcolor{darklavender}{Ciudad de México.}
\end{itemize}

\section*{Habilidades técnicas}
\skill{Lenguajes de programación}{\textsc{JavaScript},  \textsc{TypeScript},  \textsc{Golang} }
\skill{Frameworks}{\textsc{ReactJS, NextJS, Gatsby, Express, Echo}}
\skill{Frontend}{\textsc{HTML, CSS, SCSS, Tailwind, MUI}}
\skill{Bases de datos}{\textsc{MongoDB, Redis, PostgreSQL, MariaDB}}
\skill{Cloud}{\textsc{AWS, Digital Ocean, GCP, Vercel, Azure}}
\skill{Herramientas}{\textsc{Markdown, Git, Docker, Podman, \LaTeX}}
\skill{ }{\textsc{Bash, Jasper Reports, Asistentes de IA}}

\section*{Experiencia}
\work{Líder equipo web}{Ene 2020 -- Jul 2025}{Flink/Webull MX}{
\begin{itemize}[leftmargin=*, itemsep=0em, label=- ]
    \item Estandaricé el uso de NextJS y TypeScript como stack principal en todos los proyectos web, liderando la migración de sistemas legacy y su implementación en nuevos desarrollos.
    \item Diseñé soluciones web de alto rendimiento, con enfoque en la optimización de procesos y la mejora de la experiencia del usuario.
    \item Implementé pipelines de CI/CD utilizando GitLab CI, AWS ECS/ECR, Lambdas y contenedores, lo que agilizó los procesos de desarrollo y mantenimiento en producción.
    \item Diseñé e implementé APIs para proyectos web, incluyendo el modelado de esquemas y consultas optimizadas a bases de datos transaccionales.
    \item Establecí estándares de código y mejores prácticas, con documentación y configuraciones que garantizaron la calidad y consistencia del código.
    \item Gestioné proyectos con metodologías ágiles, asegurando entregas oportunas y el cumplimiento de objetivos.
    \item Recopilé y analicé los requisitos de clientes internos, diseñando flujos de trabajo, mockups y funcionalidades alineadas con las necesidades del negocio.
\end{itemize}
}
\vspace*{1em}
\work{Desarrollador Frontend}{Nov 2018 -- Ene 2020}{Forward
company}{
\begin{itemize}[leftmargin=*, itemsep=0em, label=-]
    \item Renové el UI de la plataforma LMS de la empresa al migrar el 100\% de los estilos inline a un preprocesador SCSS, lo que mejoró la mantenibilidad y la escalabilidad.
    \item Implementé carga dinámica de contenido en HTML con archivos de configuración y localización, lo que proporcionó una experiencia de usuario más personalizada.
    \item Colaboré con los stakeholders para recopilar y refinar los requisitos de la plataforma LMS, con flujos de UX intuitivos, mockups y funcionalidades que respondieron a las necesidades de los usuarios.
    \item Lideré el desarrollo y el mantenimiento de los componentes principales del frontend, asegurando la estabilidad y el rendimiento de la plataforma.
\end{itemize}}
\vspace*{1em}
\work{Consultor de Tecnología}{Nov 2018 -- Ene 2020}{Spanish Training Academy}{
\begin{itemize}[leftmargin=*, itemsep=0em, label=-]
    \item Desarrollé un sitio web informativo en WordPress alojado en un servidor Linux Ubuntu, con información sobre los cursos, precios, formularios de contacto y páginas esenciales para promocionar la oferta de la academia.
    \item Implementé y configuré una plataforma LMS Moodle en un servidor DigitalOcean, con certificados SSL, bases de datos configuradas y administración de usuarios para garantizar una entrega de cursos segura y sin problemas.
    \item Proporcioné soporte de TI y gestioné las herramientas de GSuite, con el objetivo de mantener operaciones fluidas y mejorar la productividad dentro de la organización.
\end{itemize}}

\section*{Educación}
\education{Licenciatura en Física}{2009 -- 2015}{UNAM}{}
\education{Ing. Diseño de software y redes}{2023 -- En curso}{UVM}{}

\section*{Lenguajes}
\lan{Español}{Nativo}
\lan{Inglés}{B2}

\section*{Intereses}
\begin{itemize}[leftmargin=*, itemsep=0em, label=-]
    \item Ciencias de la computación
    \item Educación
    \item Finanzas
    \item Divulgación científica
    \item Videojuegos
\end{itemize}

\end{document}
